\documentclass[12pt, titlepage]{article}

\usepackage{fullpage}
\usepackage[round]{natbib}
\usepackage{multirow}
\usepackage{booktabs}
\usepackage{tabularx}
\usepackage{graphicx}
\usepackage{float}
\usepackage{hyperref}
\hypersetup{
    colorlinks,
    citecolor=blue,
    filecolor=black,
    linkcolor=red,
    urlcolor=blue
}

\input{../../Comments}
%% Common Parts

\newcommand{\progname}{The Crazy Tens} % PUT YOUR PROGRAM NAME HERE
\newcommand{\authname}{Team \#25, The Crazy Four
\\Ruida Chen
\\Ammar Sharbat 
\\Alvin Qian
\\Jiaming Li} % AUTHOR NAMES                  

\usepackage{hyperref}
    \hypersetup{colorlinks=true, linkcolor=blue, citecolor=blue, filecolor=blue,
                urlcolor=blue, unicode=false}
    \urlstyle{same}
                                


\newcounter{acnum}
\newcommand{\actheacnum}{AC\theacnum}
\newcommand{\acref}[1]{AC\ref{#1}}

\newcounter{ucnum}
\newcommand{\uctheucnum}{UC\theucnum}
\newcommand{\uref}[1]{UC\ref{#1}}

\newcounter{mnum}
\newcommand{\mthemnum}{M\themnum}
\newcommand{\mref}[1]{M\ref{#1}}

\begin{document}

\title{Module Guide for The Crazy Four} 
\author{
    Team \#25, The Crazy Four \\[1ex]
    Ruida Chen \\
    Ammar Sharbat \\
    Alvin Qian \\
    Jiaming Li
}
\date{\today}

\maketitle

\pagenumbering{roman}

\section{Revision History}

\begin{tabularx}{\textwidth}{p{3cm}p{2cm}X}
\toprule {\bf Date} & {\bf Version} & {\bf Notes}\\
\midrule
Nov 6 & Alvin, Ammar, Ruida, Jim & Initial draft of basic MG (not in report)\\
Nov 7 & Above \& Chris Schankula & Review and feedback of basic MG\\
Nov 8 & Alvin & Module Hierarchy and decomposition\\
Nov 9 & Alvin & Updated API's for game action modules\\
Nov 13 & Alvin & modified module decomposition to remove incorrect content\\
Nov 13 & Ammar & added ACs, AC Traceability, Use Hierarchy, UI, Design of CPs; WebSocket, Timeline\\
\bottomrule
\end{tabularx}

\newpage

\section{Reference Material}

This section records information for easy reference.

\subsection{Abbreviations and Acronyms}

\renewcommand{\arraystretch}{1.2}
\begin{tabular}{l l} 
  \toprule		
  \textbf{symbol} & \textbf{description}\\
  \midrule 
  AC & Anticipated Change\\
  ACID & Atomicity, Consistency, Isolation, Durability\\
  API & Application Programming Interface\\
  CSS & Cascading Style Sheets\\
  DAG & Directed Acyclic Graph \\
  DOM & Document Object Model\\
  FR & Functional Requirement\\
  HTTP & Hypertext Transfer Protocol\\
  JSON & JavaScript Object Notation\\
  JWT & JSON Web Token\\
  M & Module \\
  MG & Module Guide \\
  NFR & Non-Functional Requirement\\
  OS & Operating System \\
  R & Requirement\\
  REST & Representational State Transfer\\
  SC & Scientific Computing \\
  SR & Safety Requirement\\
  SRS & Software Requirements Specification\\
  SQL & Structured Query Language\\
  UC & Unlikely Change \\
  UI & User Interface\\
  \bottomrule
\end{tabular}\\

\newpage

\tableofcontents

\listoftables

\listoffigures

\newpage

\pagenumbering{arabic}

\section{Introduction}

Decomposing a system into modules is a commonly accepted approach to developing
software.  A module is a work assignment for a programmer or programming
team~\citep{ParnasEtAl1984}.  We advocate a decomposition
based on the principle of information hiding~\citep{Parnas1972a}.  This
principle supports design for change, because the ``secrets'' that each module
hides represent likely future changes.  Design for change is valuable in SC,
where modifications are frequent, especially during initial development as the
solution space is explored.  

Our design follows the rules layed out by \citet{ParnasEtAl1984}, as follows:
\begin{itemize}
\item System details that are likely to change independently should be the
  secrets of separate modules.
\item Each data structure is implemented in only one module.
\item Any other program that requires information stored in a module's data
  structures must obtain it by calling access programs belonging to that module.
\end{itemize}

After completing the first stage of the design, the Software Requirements
Specification (SRS), the Module Guide (MG) is developed~\citep{ParnasEtAl1984}. The MG
specifies the modular structure of the system and is intended to allow both
designers and maintainers to easily identify the parts of the software.  The
potential readers of this document are as follows:

\begin{itemize}
\item New project members: This document can be a guide for a new project member
  to easily understand the overall structure and quickly find the
  relevant modules they are searching for.
\item Maintainers: The hierarchical structure of the module guide improves the
  maintainers' understanding when they need to make changes to the system. It is
  important for a maintainer to update the relevant sections of the document
  after changes have been made.
\item Designers: Once the module guide has been written, it can be used to
  check for consistency, feasibility, and flexibility. Designers can verify the
  system in various ways, such as consistency among modules, feasibility of the
  decomposition, and flexibility of the design.
\end{itemize}

The rest of the document is organized as follows. Section
\ref{SecChange} lists the anticipated and unlikely changes of the software
requirements. Section \ref{SecMH} summarizes the module decomposition that
was constructed according to the likely changes. Section \ref{SecConnection}
specifies the connections between the software requirements and the
modules. Section \ref{SecMD} gives a detailed description of the
modules. Section \ref{SecTM} includes two traceability matrices. One checks
the completeness of the design against the requirements provided in the SRS. The
other shows the relation between anticipated changes and the modules. Section
\ref{SecUse} describes the use relation between modules.


%%add
\section{Anticipated and Unlikely Changes} \label{SecChange}

\section{Module Hierarchy} \label{SecMH}

This section provides an overview of the module design. Modules are summarized
in a hierarchy decomposed by secrets in Table \ref{TblMH}. The modules listed
below, which are leaves in the hierarchy tree, are the modules that will
actually be implemented.


\begin{description}
% --- Backend Modules ---
\item [\refstepcounter{mnum} \mthemnum \label{mAPI}:] API Module
    \begin{itemize}
        \item Provides stateless HTTP (REST) endpoints for auth and profile management.
    \end{itemize}

\item [\refstepcounter{mnum} \mthemnum \label{mRealtimeGateway}:] Real-time Gateway Module
    \begin{itemize}
        \item Manages stateful WebSocket connections for live gameplay and state syncing.
    \end{itemize}

\item [\refstepcounter{mnum} \mthemnum \label{mMatchmaking}:] Matchmaking Module
    \begin{itemize}
        \item Handles game lobby creation, joining, and starting a match.
    \end{itemize}

\item [\refstepcounter{mnum} \mthemnum \label{mAuth}:] Authentication Module
    \begin{itemize}
        \item Manages user identity, password hashing, and session token generation.
    \end{itemize}

\item [\refstepcounter{mnum} \mthemnum \label{mRepository}:] Repository Module
    \begin{itemize}
        \item Abstracts all database queries (SQL) for creating, reading, updating, and deleting data.
    \end{itemize}

\item [\refstepcounter{mnum} \mthemnum \label{mAudit}:] Audit Module
    \begin{itemize}
        \item Logs important server-side events for debugging and security.
    \end{itemize}

% --- Frontend Modules ---
\item [\refstepcounter{mnum} \mthemnum \label{mRealtimeClient}:] Real-time Client Module
    \begin{itemize}
        \item Establishes and maintains the client-side WebSocket connection; sends/receives game events.
    \end{itemize}

\item [\refstepcounter{mnum} \mthemnum \label{mAppShell}:] Application Shell Module
    \begin{itemize}
        \item The main React component providing global layout, navigation, and state.
    \end{itemize}

\item [\refstepcounter{mnum} \mthemnum \label{mAuthClient}:] Authentication Client Module
    \begin{itemize}
        \item Provides the UI and logic for login/signup forms.
    \end{itemize}

\item [\refstepcounter{mnum} \mthemnum \label{mLobbyView}:] Lobby View Module
    \begin{itemize}
        \item UI component for displaying, creating, and joining game lobbies.
    \end{itemize}

\item [\refstepcounter{mnum} \mthemnum \label{mGameBoardView}:] Game Board View Module
    \begin{itemize}
        \item UI component that renders the main game interface (hands, deck, discard pile).
    \end{itemize}

\item [\refstepcounter{mnum} \mthemnum \label{mMoveController}:] Move Controller Module
    \begin{itemize}
        \item Manages user input (e.g., card clicks) and highlights valid moves.
    \end{itemize}

\item [\refstepcounter{mnum} \mthemnum \label{mScoreboardView}:] Scoreboard View Module
    \begin{itemize}
        \item UI component for displaying end-of-round scores in decimal and Dozenal.
    \end{itemize}

\item [\refstepcounter{mnum} \mthemnum \label{mProfileView}:] Profile View Module
    \begin{itemize}
        \item UI component for displaying user statistics and game history.
    \end{itemize}

% --- Core/Behaviour Modules ---
\item [\refstepcounter{mnum} \mthemnum \label{mGameEngine}:] Game Engine Module
    \begin{itemize}
        \item Manages the core game state (deck, hands) and turn progression.
    \end{itemize}

\item [\refstepcounter{mnum} \mthemnum \label{mRules}:] Rules Module
    \begin{itemize}
        \item Stateless logic to validate moves (e.g., match suit, rank, or Dozenal sum).
    \end{itemize}

\item [\refstepcounter{mnum} \mthemnum \label{mScoring}:] Scoring Module
    \begin{itemize}
        \item Calculates scores at the end of a round.
    \end{itemize}

\item [\refstepcounter{mnum} \mthemnum \label{mBaseConversion}:] Base Conversion Module
    \begin{itemize}
        \item Utility to convert numbers between decimal and Dozenal.
    \end{itemize}

\item [\refstepcounter{mnum} \mthemnum \label{mGameActions}:] Game Actions Module
    \begin{itemize}
        \item Defines types and structure for player actions (play card, draw, declare suit, submit score tally).
    \end{itemize}

% --- Hardware Hiding ---
\item [\refstepcounter{mnum} \mthemnum \label{mOS}:] Operating System Module
    \begin{itemize}
        \item Represents the server's OS, providing the Node.js runtime environment.
    \end{itemize}

\item [\refstepcounter{mnum} \mthemnum \label{mBrowser}:] Browser Runtime Module
    \begin{itemize}
        \item Represents the client's web browser, providing the React runtime environment.
    \end{itemize}

\item [\refstepcounter{mnum} \mthemnum \label{mDB}:] Database Module
    \begin{itemize}
        \item Represents the PostgreSQL software that handles physical data storage.
    \end{itemize}
\end{description}

\begin{table}[H]
\centering
\begin{tabular}{p{0.3\textwidth} p{0.3\textwidth} p{0.3\textwidth}}
\toprule
\textbf{Level 1} & \textbf{Level 2} & \textbf{Level 3 (Leaf Modules)}\\
\midrule

\multirow{3}{0.3\textwidth}{Hardware-Hiding Module} & ~ & \mref{mOS} (Server OS) \\
& ~ & \mref{mBrowser} (Client Runtime) \\
& ~ & \mref{mDB} (PostgreSQL) \\
\midrule

\multirow{5}{0.3\textwidth}{Behaviour-Hiding Module} & (Core Domain Logic) & \mref{mGameEngine} \\
& & \mref{mRules} \\
& & \mref{mScoring} \\
& & \mref{mBaseConversion} \\
& & \mref{mGameActions} \\
\midrule

\multirow{15}{0.3\textwidth}{Software Decision Module} & \multirow{6}{0.3\textwidth}{Backend (Server)} & \mref{mAPI} \\
& & \mref{mRealtimeGateway} \\
& & \mref{mMatchmaking} \\
& & \mref{mAuth} \\
& & \mref{mRepository} \\
& & \mref{mAudit} \\
\cmidrule{2-3}
& \multirow{9}{0.3\textwidth}{Frontend (Client)} & \mref{mRealtimeClient} \\
& & \mref{mAppShell} \\
& & \mref{mAuthClient} \\
& & \mref{mLobbyView} \\
& & \mref{mGameBoardView} \\
& & \mref{mMoveController} \\
& & \mref{mScoreboardView} \\
& & \mref{mProfileView} \\
\bottomrule

\end{tabular}
\caption{Module Hierarchy}
\label{TblMH}
\end{table}

\section{Connection Between Requirements and Design} \label{SecConnection}

The design of the system is intended to satisfy the requirements developed in
the SRS. In this stage, the system is decomposed into modules. The connection
between requirements and modules is listed in Table~\ref{TblRT}.\\


The design of the system is intended to satisfy the requirements developed in
the SRS. In this stage, the system is decomposed into modules. The connection
between requirements and modules is listed in the Traceability Matrix in Section \ref{SecTM} (Table \ref{TblRT}). This decomposition ensures that each Functional Requirement (FR), Non-functional Requirement (NFR), and Safety Requirement (SR) has a clear owner in the design, facilitating implementation and verification.

For example, core gameplay logic (FR-1 to FR-5) is satisfied by the \mref{mGameEngine} and \mref{mRules} modules, while the user-facing presentation (FR-7, FR-9) is handled by frontend modules like \mref{mScoreboardView} and \mref{mGameBoardView}. Security and data persistence requirements (FR-10 to FR-17, SR-3, SR-8) are satisfied by the backend's \mref{mAuth} and \mref{mRepository} modules.

\section{Module Decomposition} \label{SecMD}

Modules are decomposed according to the principle of ``information hiding''
proposed by \citet{ParnasEtAl1984}. The \emph{Secrets} field in a module
decomposition is a brief statement of the design decision hidden by the
module. The \emph{Services} field specifies \emph{what} the module will do
without documenting \emph{how} to do it. For each module, a suggestion for the
implementing software is given under the \emph{Implemented By} title.

Only the leaf modules in the hierarchy have to be implemented.

\subsection{Hardware Hiding Modules}

\subsubsection{Operating System Module (\mref{mOS})}
\begin{quote}
\begin{description}
  \item[Secrets:] Process scheduling, filesystem, Node.js runtime environment, network stack.
  \item[Services:] Provides the execution environment for the backend server.
  \item[Implemented By:] OS (such as Linux)
  \item[Type of Module:] Hardware
\end{description}
\end{quote}

\subsubsection{Browser Runtime Module (\mref{mBrowser})}
\begin{quote}
\begin{description}
  \item[Secrets:] DOM rendering, event loop, TypeScript (React) execution, WebSocket/HTTP client implementation.
  \item[Services:] Provides the execution environment for the frontend client.
  \item[Implemented By:] Browser (Chrome, Firefox, Edge)
  \item[Type of Module:] Hardware
\end{description}
\end{quote}

\subsubsection{Database Module (\mref{mDB})}
\begin{quote}
\begin{description}
  \item[Secrets:] Data storage on disk, indexing, transaction (ACID) implementation, SQL query optimization.
  \item[Services:] Provides persistent storage for user and game data.
  \item[Implemented By:] PostgreSQL
  \item[Type of Module:] Hardware
\end{description}
\end{quote}

\subsection{Behaviour-Hiding Module}

\subsubsection{Game Engine Module (\mref{mGameEngine})}
\begin{quote}
\begin{description}
  \item[Secrets:] Internal representations of the cards, deck, discard pile, players, and the turn-management state machine.
  \item[Services:] Creates new matches, enforces turn order, applies validated moves, and determines when a round or match is finished.
  \item[Implemented By:] The Crazy Four (TypeScript)
  \item[Type of Module:] Abstract Object
\end{description}
\end{quote}

\subsubsection{Rules Module (\mref{mRules})}
\begin{quote}
\begin{description}
  \item[Secrets:] Exact move-validation criteria, including how matching ranks, suits, dozenal sums, and special cards are handled.
  \item[Services:] Confirms whether a proposed move is legal and enumerates valid moves for a player based on the current game situation.
  \item[Implemented By:] The Crazy Four (TypeScript)
  \item[Type of Module:] Abstract Object
\end{description}
\end{quote}

\subsubsection{Scoring Module (\mref{mScoring})}
\begin{quote}
\begin{description}
  \item[Secrets:] The scoring equation that converts remaining cards into round points and aggregates them over a match.
  \item[Services:] Produces the score summary for each player whenever a round ends.
  \item[Implemented By:] The Crazy Four (TypeScript)
  \item[Type of Module:] Abstract Object
\end{description}
\end{quote}

\subsubsection{Base Conversion Module (\mref{mBaseConversion})}
\begin{quote}
\begin{description}
  \item[Secrets:] The mapping of digits and symbols used to move between decimal and dozenal numbers.
  \item[Services:] Translates numeric values to and from dozenal form for scoring logic and UI presentation.
  \item[Implemented By:] The Crazy Four (TypeScript)
  \item[Type of Module:] Abstract Data Type
\end{description}
\end{quote}

\subsubsection{Game Actions Module (\mref{mGameActions})}
\begin{quote}
\begin{description}
  \item[Secrets:] The canonical data structures and serialization format that represent every move a player can make.
  \item[Services:] Defines and validates the action payloads that flow between client and server, keeping both sides in sync on network contracts.
  \item[Implemented By:] The Crazy Four (TypeScript shared library)
  \item[Type of Module:] Abstract Data Type
\end{description}
\end{quote}

\subsection{Software Decision Module - Backend}

\subsubsection{API Module (\mref{mAPI})}
\begin{quote}
\begin{description}
  \item[Secrets:] REST endpoint structure, payload schemas, and HTTP conventions for every backend capability.
  \item[Services:] Exposes stateless HTTP routes for authentication, profile management, and bootstrapping new games as defined in the SRS.
  \item[Implemented By:] The Crazy Four (Node.js, Express)
  \item[Type of Module:] Abstract Object
\end{description}
\end{quote}

\subsubsection{Real-time Gateway Module (\mref{mRealtimeGateway})}
\begin{quote}
\begin{description}
  \item[Secrets:] Real-time messaging strategy, room management, and conflict-resolution logic for server-authoritative play.
  \item[Services:] Hosts WebSocket connections, validates incoming moves, and pushes synchronized game state updates to every participant.
  \item[Implemented By:] The Crazy Four (Node.js, Socket.io)
  \item[Type of Module:] Abstract Object
\end{description}
\end{quote}

\subsubsection{Matchmaking Module (\mref{mMatchmaking})}
\begin{quote}
\begin{description}
  \item[Secrets:] The pairing heuristics, lobby data structures, and invitation policies for assembling tables.
  \item[Services:] Creates, lists, and manages lobbies so players can host private games or enter matchmaking queues.
  \item[Implemented By:] The Crazy Four (Node.js)
  \item[Type of Module:] Abstract Object
\end{description}
\end{quote}

\subsubsection{Authentication Module (\mref{mAuth})}
\begin{quote}
\begin{description}
  \item[Secrets:] Password hashing configuration, credential storage details, and token-signing keys.
  \item[Services:] Creates accounts, validates logins, manages guest sessions, and issues/verifies tokens used by the rest of the backend.
  \item[Implemented By:] The Crazy Four (Node.js)
  \item[Type of Module:] Abstract Object
\end{description}
\end{quote}

\subsubsection{Repository Module (\mref{mRepository})}
\begin{quote}
\begin{description}
  \item[Secrets:] Schema design, optimized SQL queries, and database access strategies.
  \item[Services:] Offers a clean persistence interface for storing players, credentials, match history, and statistics while shielding callers from database details.
  \item[Implemented By:] The Crazy Four (Node.js, node-postgres)
  \item[Type of Module:] Abstract Data Type
\end{description}
\end{quote}

\subsubsection{Audit Module (\mref{mAudit})}
\begin{quote}
\begin{description}
  \item[Secrets:] The exact event schema, retention policy, and storage targets for operational logs.
  \item[Services:] Captures authentication, gameplay, and system events to support debugging, compliance, and user inquiries.
  \item[Implemented By:] The Crazy Four (Node.js, Winston)
  \item[Type of Module:] Abstract Object
\end{description}
\end{quote}

\subsection{Software Decision Module - Frontend}

\subsubsection{Real-time Client Module (\mref{mRealtimeClient})}
\begin{quote}
\begin{description}
  \item[Secrets:] Connection lifecycle logic, buffering strategy, and reconnection heuristics for the browser client.
  \item[Services:] Establishes WebSocket links to the \mref{mRealtimeGateway}, relays user actions, and applies server updates to the local UI state.
  \item[Implemented By:] The Crazy Four (TypeScript, Socket.io-client)
  \item[Type of Module:] Abstract Object
\end{description}
\end{quote}

\subsubsection{Application Shell Module (\mref{mAppShell})}
\begin{quote}
\begin{description}
  \item[Secrets:] App-wide navigation plan, shared layout primitives, and global state wiring (theme, auth awareness).
  \item[Services:] Hosts the consistent chrome of the site and orchestrates routing between major views.
  \item[Implemented By:] The Crazy Four (React)
  \item[Type of Module:] Abstract Object
\end{description}
\end{quote}

\subsubsection{Authentication Client Module (\mref{mAuthClient})}
\begin{quote}
\begin{description}
  \item[Secrets:] Decisions about secure token storage and the flows for refreshing or clearing credentials in the browser.
  \item[Services:] Presents login, signup, and logout experiences while coordinating with the \mref{mAPI} for authentication calls.
  \item[Implemented By:] The Crazy Four (React)
  \item[Type of Module:] Abstract Object
\end{description}
\end{quote}

\subsubsection{Lobby View Module (\mref{mLobbyView})}
\begin{quote}
\begin{description}
  \item[Secrets:] Layout, styling, and interaction patterns for discovering or hosting lobbies.
  \item[Services:] Shows available rooms, lets players create or join sessions, and triggers matchmaking calls via \mref{mAPI} and \mref{mRealtimeClient}.
  \item[Implemented By:] The Crazy Four (React)
  \item[Type of Module:] Abstract Object
\end{description}
\end{quote}

\subsubsection{Game Board View Module (\mref{mGameBoardView})}
\begin{quote}
\begin{description}
  \item[Secrets:] Visual composition of the board, card animations, and responsive behavior across devices.
  \item[Services:] Draws the playable surface, displays player hands and discard piles, and highlights valid actions per the rule engine.
  \item[Implemented By:] The Crazy Four (React)
  \item[Type of Module:] Abstract Object
\end{description}
\end{quote}

\subsubsection{Move Controller Module (\mref{mMoveController})}
\begin{quote}
\begin{description}
  \item[Secrets:] Gesture-handling patterns and UX rules for how players select cards or declare suits.
  \item[Services:] Interprets user intent on the board, performs light validation, and forwards structured actions through the \mref{mRealtimeClient}.
  \item[Implemented By:] The Crazy Four (React hooks and event handlers)
  \item[Type of Module:] Abstract Object
\end{description}
\end{quote}

\subsubsection{Scoreboard View Module (\mref{mScoreboardView})}
\begin{quote}
\begin{description}
  \item[Secrets:] Presentation choices for multi-base score displays and animations for round summaries.
  \item[Services:] Shows standings after each round, presenting both decimal and dozenal scores in a clear, accessible format.
  \item[Implemented By:] The Crazy Four (React)
  \item[Type of Module:] Abstract Object
\end{description}
\end{quote}

\subsubsection{Profile View Module (\mref{mProfileView})}
\begin{quote}
\begin{description}
  \item[Secrets:] Layout decisions for profile cards, statistics summaries, and sensitive account actions.
  \item[Services:] Displays player history, stats, and account-management controls, including export or deletion requests tied to FR-15..17.
  \item[Implemented By:] The Crazy Four (React)
  \item[Type of Module:] Abstract Object
\end{description}
\end{quote}

\section{Traceability Matrix} \label{SecTM}

This section shows two traceability matrices: between the modules and the
requirements and between the modules and the anticipated changes.

% the table should use mref, the requirements should be named, use something
% like fref
\begin{table}[H]
\centering
\caption{Trace Between Requirements and Modules (TblRT)}
\label{TblRT}
\begin{tabularx}{\textwidth}{lX}
\toprule
\textbf{Requirement (FR/NFR/SR)} & \textbf{Primary Modules}\\
\midrule
FR-1 Start new game & \mref{mAPI}, \mref{mMatchmaking}, \mref{mGameEngine}, \mref{mRepository}\\
FR-2 Turn management & \mref{mGameEngine}, \mref{mRules}, \mref{mGameActions}, \mref{mRealtimeGateway}, \mref{mRealtimeClient}, \mref{mMoveController}, \mref{mGameBoardView}\\
FR-3 Rule validation & \mref{mRules}, \mref{mGameEngine}, \mref{mGameActions}, \mref{mMoveController}, \mref{mGameBoardView}\\
FR-4 Special cards & \mref{mRules}, \mref{mGameEngine}, \mref{mGameActions}, \mref{mMoveController}, \mref{mGameBoardView}\\
FR-5 End of game & \mref{mGameEngine}, \mref{mScoring}, \mref{mScoreboardView}, \mref{mRepository}\\
FR-6 Calculate score & \mref{mScoring}, \mref{mBaseConversion}, \mref{mGameActions}, \mref{mScoreboardView}\\
FR-7 Display score & \mref{mScoreboardView}, \mref{mBaseConversion}\\
FR-9 Highlight valid moves & \mref{mMoveController}, \mref{mGameBoardView}, \mref{mRules}\\
FR-10 Account creation & \mref{mAPI}, \mref{mAuth}, \mref{mRepository}, \mref{mAuthClient}\\
FR-11 Login or Logout & \mref{mAPI}, \mref{mAuth}, \mref{mRepository}, \mref{mAuthClient}\\
FR-12 Guest mode & \mref{mAPI}, \mref{mAuth}, \mref{mAuthClient}\\
FR-13 Credential validation & \mref{mAuth}, \mref{mAPI}, \mref{mRepository}\\
FR-14 Data storage & \mref{mRepository}, \mref{mAPI}, \mref{mAudit}\\
FR-15 Data retrieval & \mref{mRepository}, \mref{mAPI}, \mref{mProfileView}\\
FR-16 Data update & \mref{mRepository}, \mref{mAPI}, \mref{mProfileView}\\
FR-17 Data deletion & \mref{mRepository}, \mref{mAPI}, \mref{mProfileView}\\
\midrule
NFR (Performance) & \mref{mRealtimeGateway}, \mref{mGameEngine}, \mref{mRealtimeClient}, \mref{mGameBoardView}\\
NFR (Usability) & \mref{mGameBoardView}, \mref{mAppShell}\\
NFR (Robustness) & \mref{mRealtimeClient}, \mref{mRealtimeGateway}, \mref{mRepository}\\
NFR (Maintainability) & \mref{mRules}, \mref{mScoring}, \mref{mAPI}, \mref{mAudit}\\
\midrule
SR-1 (Dozenal validation) & \mref{mRules}, \mref{mBaseConversion}, \mref{mGameEngine}\\
SR-2 (UI feedback) & \mref{mGameBoardView}, \mref{mAppShell}\\
SR-3 (Data persistence) & \mref{mRepository}, \mref{mAPI}\\
SR-4 (Accurate scoring) & \mref{mScoring}, \mref{mBaseConversion}\\
SR-5 (Session recovery) & \mref{mRealtimeClient}, \mref{mRealtimeGateway}, \mref{mAuth}\\
SR-7 (Encrypted transmit) & (Hardware-Hiding: TLS Layer), \mref{mAPI}, \mref{mRealtimeGateway}\\
SR-8 (Secure storage) & \mref{mRepository}, \mref{mAuth} \\
SR-10 (Input validation) & \mref{mAPI}, \mref{mRealtimeGateway}, \mref{mMoveController}\\
\bottomrule
\end{tabularx}
\end{table}


%%add

\begin{table}[H]
\centering
\caption{Trace Between Anticipated Changes and Modules (TblACT)}
\label{TblACT}
\begin{tabularx}{\textwidth}{p{0.2\textwidth} X}
\toprule
\textbf{AC} & \textbf{Modules}\\
\midrule
\acref{acRules} & \mref{mRules}, \mref{mGameEngine} \\
\acref{acBases} & \mref{mBaseConversion}, \mref{mScoring}, \mref{mRules}, \mref{mScoreboardView} \\
\acref{acScoring} & \mref{mScoring}, \mref{mScoreboardView} \\
\acref{acUI} & \mref{mAppShell}, \mref{mGameBoardView}, \mref{mScoreboardView}, \mref{mLobbyView}, \mref{mProfileView} \\
\acref{acSchema} & \mref{mRepository}, \mref{mAPI}, \mref{mProfileView} \\
\acref{acRealtime} & \mref{mRealtimeGateway}, \mref{mRealtimeClient} \\
\acref{acAuth} & \mref{mAuth}, \mref{mAuthClient}, \mref{mAPI} \\
\bottomrule
\end{tabularx}
\end{table}


\section{Use Hierarchy} \label{SecUse}
%\includegraphics[width=0.7\textwidth]{UsesHierarchy.png}



%%add

\section{User Interfaces}

%%add

\section{Design of Communication Protocols}

%%add

\section{Timeline}

%%add

\newpage

\bibliographystyle {plainnat}
\bibliography{../../../refs/References}

\end{document}
