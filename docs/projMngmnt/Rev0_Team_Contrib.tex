\documentclass{article}

\usepackage{float}
\restylefloat{table}

\usepackage{booktabs}

\usepackage{pgf}

\title{Team Contributions: Rev 0\\\progname}

\author{\authname}

\date{}

\input{../Comments}
%% Common Parts

\newcommand{\progname}{The Crazy Tens} % PUT YOUR PROGRAM NAME HERE
\newcommand{\authname}{Team \#25, The Crazy Four
\\Ruida Chen
\\Ammar Sharbat 
\\Alvin Qian
\\Jiaming Li} % AUTHOR NAMES                  

\usepackage{hyperref}
    \hypersetup{colorlinks=true, linkcolor=blue, citecolor=blue, filecolor=blue,
                urlcolor=blue, unicode=false}
    \urlstyle{same}
                                


\begin{document}

\maketitle

This document summarizes the contributions of each team member for the Rev 0
Demo.  The time period of interest is the time between the PoC demo for Team 25 (Date: November 26, 2025)
and the Rev 0 demo (Date: February 02, 2026); the contributions prior to the PoC are NOT included in ANY section.

\section{Demo Plans}

\subsection*{Overview}
The Rev 0 demo will be performed in-person and run locally from a group member's laptop. The goal is to demonstrate a complete two-player game flow with authentication, lobby management, and real-time gameplay via WebSocket communication.

\subsection*{Setup}
\begin{itemize}
    \item Start the backend server (\texttt{Node.js} with Express and Socket.IO) on the presenter's machine.
    \item Start the frontend development server (\texttt{Vite + React}) on the presenter's machine.
    \item Open the game in two separate browser windows (one normal, one incognito) to simulate two players.
\end{itemize}

\subsection*{Demo Flow (approx. 8--10 minutes)}
\begin{enumerate}
    \item Brief introduction of demo objectives and improvements since POC (30s).
    \item Demonstrate user authentication:
        \begin{itemize}
            \item Register/login for Player 1 in the normal browser window.
            \item Register/login for Player 2 in the incognito window.
        \end{itemize}
    \item Demonstrate lobby system:
        \begin{itemize}
            \item Player 1 creates a lobby and copies the lobby ID.
            \item Player 2 joins the lobby using the shared lobby ID.
            \item Player 1 selects numeral system (Dozenal/Decimal) and starts the match.
        \end{itemize}
    \item Show automatic transition from Lobby screen to Game screen for both players.
    \item Demonstrate real-time gameplay with the new UI:
        \begin{itemize}
            \item Show opponent's face-down hand and card count.
            \item Playing cards from visible hand with playability highlighting.
            \item Demonstrate wildcard (10) with suit picker modal and golden card styling.
            \item Demonstrate skip card (6) granting free play with blue card styling.
            \item Drawing cards when no valid plays are available.
            \item Show real-time score updates and turn indicators.
        \end{itemize}
    \item Demonstrate the Decimal/Dozenal display toggle in the game header.
    \item Drive the game toward an endgame state and show round/game completion.
    \item Conclude with current limitations and planned improvements for Rev 1, then take questions.
\end{enumerate}

\subsection*{Notes}
\begin{itemize}
    \item Authentication is now functional with JWT tokens stored in localStorage.
    \item WebSocket handles real-time game state synchronization between players.
    \item Will mention current limitations (local deployment only, basic error handling) and planned next steps (deployment, improved UI/UX, additional game modes).
\end{itemize}

\section{Team Meeting Attendance}

\begin{table}[H]
\centering
\begin{tabular}{ll}
\toprule
\textbf{Student} & \textbf{Meetings}\\
\midrule
Total & 2 \\
Ruida Chen & 2 \\
Jiaming Li & 2 \\
Alvin Qian & 2 \\
Ammar Sharbat & 2 \\
\bottomrule
\end{tabular}
\end{table}

Explanation:
\begin{itemize}
    \item We don't meet that often as a team, rather we use Discord for communication generally, which is very hard to trace.
    \item All team meetings we have had to date have been added our Github Repo. Here is our meeting \href{https://github.com/The-Crazy-Four-Games/Crazy-Tens-Game/issues/213}{last November}, and \href{https://github.com/The-Crazy-Four-Games/Crazy-Tens-Game/issues/163}{this January}.
\end{itemize}

\section{Supervisor/Stakeholder Meeting Attendance}

\noindent \textbf{Supervisor's Name: } Paul Rapoport; Email: rapoport@mcmaster.ca\\
\noindent \textbf{Other Stakeholders: } Card game and web game players, like Teammate Ammar's parents.

\begin{table}[H]
\centering
\begin{tabular}{ll}
\toprule
\textbf{Student} & \textbf{Meetings}\\
\midrule
Total & 5 \\
Ruida Chen & 2 \\
Jiaming Li & 2 \\
Alvin Qian & 3 \\
Ammar Sharbat & 4 \\
\bottomrule
\end{tabular}
\end{table}

Explanation:
\begin{itemize}
    \item All Supervisor meetings have been added to GitHub for Traceability. See the Issue for the Supervisor Meeting section of this report \href{https://github.com/The-Crazy-Four-Games/Crazy-Eights-Game/issues/216}{here}.
    \item All meetings with Professor Rapoport were initiated by teammate Ammar Sharbat, because he is the team liaison to the supervisor.
    \item Teammate Ammar and Teammate Alvin also took an interest in Game Mechanics and Game UI respectively, and met with stakeholders to work on these issues. Ammar played the game with his parents to figure out final game mechanics, and Alvin tested out an in game feature with potential stakeholders.
\end{itemize}

\section{Lecture + Tutorial Attendance\\    or  \\Lectures + Tutorials Read}

\begin{table}[H]
\centering
\begin{tabular}{ll}
\toprule
\textbf{Student} & \textbf{Lectures}\\
\midrule
Total & 1 \\
Ruida Chen & 1 \\
Jiaming Li & 1 \\
Alvin Qian & 1 \\
Ammar Sharbat & 1 \\
\bottomrule
\end{tabular}
\end{table}

\begin{table}[H]
\centering
\begin{tabular}{ll}
\toprule
\textbf{Student} & \textbf{Tutorials}\\
\midrule
Total & 1 \\
Ruida Chen & 1 \\
Jiaming Li & 0 \\
Alvin Qian & 0 \\
Ammar Sharbat & 0 \\
\bottomrule
\end{tabular}
\end{table}

Explanation:
\begin{itemize}
    \item Note: Both tallies are not entirely accurate, because outside of Teammate Ammar and Teammate Ruida (for some), teammates have not checkmarked their attendance/reading of classes, so the number listed is are just based on teammate Ammar's best estimates of the project group's "attendance".
    \item An issue exists in our \href{https://github.com/The-Crazy-Four-Games/Crazy-Tens-Game/issues?q=lecture\%20-}{GitHub Repository} for every lecture and tutorial class for the course.
    \item This Team Contribution Report is only concerned our attendance of the \href{https://github.com/The-Crazy-Four-Games/Crazy-Tens-Game/issues/169}{lecture} and \href{https://github.com/The-Crazy-Four-Games/Crazy-Tens-Game/issues/231}{tutorial} this January, as these were the only classes since the POC Demo last November.
\end{itemize}

\section{TA Document Discussion Attendance}

\noindent \textbf{TA's Name: } [Chris Schankula]

\begin{table}[H]
\centering
\begin{tabular}{ll}
\toprule
\textbf{Student/TA} & \textbf{Meeting Attended}\\
\midrule
Total & 1\\
TA Chris Schankula & 0 \\
Ruida Chen & 0 \\
Jiaming Li & 0 \\
Alvin Qian & 0 \\
Ammar Sharbat & 0 \\
\bottomrule
\end{tabular}
\end{table}

Explanation:
\begin{itemize}
    \item TA Chris was not available to meet for the most recent Document Discussion on the \textbf{DesDoc\_Rev0} Deliverable (Due Jan 21 2026). Teammate Ammar reached out by Teams message (on Jan 19) but received no response.
    \item Otherwise, our team has attended all other TA document discussions, and TA Chris has been very helpful and provided great feedback on our deliverables.
\end{itemize}

\section{Commits}

Time Period : November 26 (POC Demo) - February 02 (Rev0 Demo)

% Define commit counts (not printed)
% If there are any updates to your commit totals, please update the variables below with the correct number.
\pgfmathtruncatemacro{\cRuida}{11}
\pgfmathtruncatemacro{\cJiaming}{2}
\pgfmathtruncatemacro{\cAmmar}{7}
\pgfmathtruncatemacro{\cAlvin}{4}
\pgfmathtruncatemacro{\cTotal}{\cRuida+\cJiaming+\cAmmar+\cAlvin}

% Compute percentages (one decimal, not printed)
\pgfmathsetmacro{\pRuida}{100*\cRuida/\cTotal}
\pgfmathsetmacro{\pJiaming}{100*\cJiaming/\cTotal}
\pgfmathsetmacro{\pAmmar}{100*\cAmmar/\cTotal}
\pgfmathsetmacro{\pAlvin}{100*\cAlvin/\cTotal}

\begin{table}[H]
\centering
\begin{tabular}{lll}
\toprule
\textbf{Student} & \textbf{Commits} & \textbf{Percent}\\
\midrule
Total & \cTotal & 100\% \\
Ruida Chen & \cRuida & \pRuida\% \\
Jiaming Li & \cJiaming & \pJiaming\% \\
Ammar Sharbat & \cAmmar & \pAmmar\% \\
Alvin Qian & \cAlvin & \pAlvin\% \\
\bottomrule
\end{tabular}
\end{table}

\textbf{Note: Different teammates use different commit styles.} Some prefer batching changes into a few large commits after completing a section, while others commit incrementally. As a result, the number of commits does not necessarily correspond directly to workload or contribution size.

\section{Issue Tracker}

Time Period : November 26 (POC Demo) - February 02 (Rev0 Demo)

\begin{table}[H]
\centering
\begin{tabular}{lll}
\toprule
\textbf{Student} & \textbf{Authored (O+C)} & \textbf{Assigned (C only)}\\
\midrule
Ruida Chen & 13 & 12 \\
Jiaming Li & 9 & 14\\
Alvin Qian & 16 & 17\\
Ammar Sharbat & 29 & 20\\
\bottomrule
\end{tabular}
\end{table}

Some issues were created long ago (from before POC Demo) but were only closed recently, either due to delay in resolution or due to being left open unintentionally.\\
Also, some issues for events/work that was finished long ago (before POC) were just made now for tracking / traceability purposes.\\
Both of these circumstances relate to legacy issues (before POC Demo) and so they do not reflect work for the current reporting period (i.e. between POC and Rev0).\\
\textbf{Rest assured, legacy issues (explained above) have been excluded from issue counts and totals in the above table.}


\section{CICD}

The project repository is hosted on GitHub and uses GitHub Actions for Continuous Integration and Continuous Deployment (CICD).
Each push or pull request triggers an automated workflow that performs the following tasks:

\begin{itemize}
    \item \textbf{Build and Lint:} The workflow installs all dependencies, compiles the code, and runs ESLint to enforce consistent formatting and syntax.
    \item \textbf{Unit Testing:} All Jest test suites are executed automatically. Code coverage reports are uploaded to Codecov.
    \item \textbf{Static Analysis:} CodeQL is run to detect potential vulnerabilities and logic errors.
    \item \textbf{Artifact Packaging:} For successful builds, the workflow produces a testable web or desktop artifact for internal review.
\end{itemize}

This setup ensures that any code merged into the \texttt{main} branch has passed validation for correctness, maintainability, and security.
By automating these checks, CICD reduces integration errors and accelerates the development feedback cycle.

\section{Team Charter Trigger Items}

The team has identified several triggers within the team charter to monitor collaboration and performance consistency:

\begin{itemize}
    \item \textbf{Commit Frequency:} Each member should contribute at least one meaningful commit per week.
    Falling below this threshold for two consecutive weeks triggers a discussion about workload balance.
    \item \textbf{Meeting Attendance:} Missing two consecutive team meetings without prior notice triggers a check-in with the member to identify scheduling or communication issues.
    \item \textbf{Branch Discipline:} All code changes must go through a pull request reviewed by at least one teammate.
    Direct commits to \texttt{main} are not allowed and will trigger an immediate process review.
    \item \textbf{Responsiveness:} Team members are expected to reply to key project communications (e.g., PR reviews or Slack updates) within 24 hours.
    Failure to respond repeatedly triggers a group discussion for reassigning responsibilities.
\end{itemize}

So far, no major trigger violations have occurred. The team has maintained consistent communication and review discipline.
If violations are observed in the future, the plan is to (1) hold a brief retrospective discussion,
(2) revise or clarify the trigger threshold if needed, and (3) document the agreed corrective action in the next meeting notes.



\section{Additional Productivity Metrics}

\wss{If your team has additional metrics of productivity, please feel free to
add them to this report.}

\end{document}